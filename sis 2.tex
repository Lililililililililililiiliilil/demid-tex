\documentclass[a4paper]{article}


\usepackage{graphicx}
\graphicspath{}
\DeclareGraphicsExtensions{.pdf,.png,.jpg}
\usepackage{caption}


\usepackage{tikz}  
\usetikzlibrary{graphs}
\usetikzlibrary{positioning,arrows}
\usetikzlibrary{trees}
\usepackage[unicode=true, colorlinks=true, linkcolor=blue, urlcolor=blue]{hyperref}

\usepackage{header}

\usepackage{enumitem}
\setlist{nolistsep, itemsep=0.1cm,parsep=0pt}

%Russian-specific packages
%--------------------------------------
\usepackage[T2A]{fontenc}
\usepackage[utf8]{inputenc}
\usepackage[russian]{babel}
\usepackage{datetime}

%--------------------------------------

%Hyphenation rules
%--------------------------------------
\usepackage{hyphenat}
\hyphenation{ма-те-ма-ти-ка вос-ста-нав-ли-вать}
\renewcommand{\dateseparator}{.}
%--------------------------------------


\title{\Huge Теория систем и системный анализ}
\author{Нечаев Иван | \href{https://t.me/mycherij}{telegram}}
\date{Версия от {\ddmmyyyydate\today} \currenttime}

\begin{document}
	\maketitle
	
	
	
	\tableofcontents
	
	\newpage
	
	\section{Введение в теорию систем.}
	
	Теория систем - метанаука, охватываем многие другие науки.
	
	Категориальное мышление - мышление в рамках разделения на абстрактные категории. С помощью него мы анализируем реальность и моделируем её.
	
	\textbf{Примеры категорий:}
	\begin{enumerate}
		\item Цвета
		\item Музыкальные ноты
		\item Фонетические звуки
		\item Буквы алфавита
		\item Начертания цифр
		\item Категории количества
		\item Категории модальности
		\item Любая классификация
		\item Субъект-объектное подразделение
	\end{enumerate}

	\textbf{Особенности и недостатки категориального мышления:}
	\begin{enumerate}
		\item Склонность нашего мозга к категориальному мышлению и желание видеть категории даже там, где их нет
		\item Склонность к придумыванию новых и противопоставлению известных категорий
		\item Склонность категориального мышления к догматике и мыслительному застою
		\item Категориальное мышление мешает видеть непрерывный спектр явлений и событий
		\item Трудности исследования явлений в непрерывном времени и вообще непрерывных величин
	\end{enumerate}

	\textbf{Редукция} - принцип познания, предполагающий сведение более сложного явления к более простым и легким для объяснения.
	
	\textbf{Редукционизм} - методологический принцип, согласно которому сложные явления могут быть полностью объяснены с помощью законов, свойственных более простым явлениям.
	
	Идеи редукционизма восходят к идеям Рене Декарта. 
	
	Концепция животных и человека как механических автоматов.
	\begin{itemize}
		\item Ходячие статуи Дедала в Афинах(легенды)
		\item Летающий деревянный голубь Архита Тарентского (lV век до н.э.)
		\item Автоматонны Вокансона(1738)
		\item Современные роботы Boston Dynamics (с 2005)
	\end{itemize}
	
	\textbf{Бритва Оккама} - "Если два объяснения одинаково хорошо объясняют явление, то стоит предпочесть более простое объяснение"
	
	Принцип получил название по имени монаха-философа Уильяма Оккама.
	
\textbf{	Проблемы, связанные с редукционизмом:}
		\begin{itemize}
		\item Ненадежность редукционизма при изучении сложных объектов и явлений
		\item Склонность редукционизма к слишком простым и часто неверным объяснениям
		\item Привычка нашего мозга к редукционизму мешает выходить за рамки и понимать сложные явления
		\item Накопление в разных науках(квантовая физика, биология, антропология, социология) явления, плохо укладывающихся в редукционистскую методологию познания
	\end{itemize}
	
	С именем Аристотеля также связано первым использование термина "система" в смысле нашей науки.
	
	\textbf{Характер и особенности теории систем:}
		\begin{itemize}
		\item Междисциплинарная наука
		\item Мета-наука
		\item Молодая наука с неустоявшейся терминологией
		\item Объединяет в  себе математические и гуманитарные методы исследования и описания различных систем
		\item Трудности прикладного использования из-за чрезвычайно широкого предмета
		\item Набор близких направлений: исследование операций, системный анализ, синергетика и др.
	\end{itemize}

\begin{figure}[h]
	\centering
	\begin{tikzpicture}[node distance=1cm, auto]  
		\tikzset{
			mynode/.style={rectangle,rounded corners,draw=black, top color=white, bottom color=white!50,very thick, inner sep=1em, minimum size=2em, text centered},
			myarrow/.style={->, >=latex', shorten >=1pt, thick},
			mylabel/.style={text width=7em, text centered} 
		}  
		\node[mynode] (manufacturer) {Общая теория систем(в широком смысле)};  
		\node[below=1.5cm of manufacturer] (dummy) {}; 
	\node[mynode, left=of dummy] (retailer1) {Теоритическая составляющая};  
	\node[mynode, right=of dummy] (retailer2) {Прикладная составляющая};
	\draw[myarrow] (manufacturer.south) -- (retailer1.north);	
	\draw[myarrow] (manufacturer.south) --  (retailer2.north);
	\node[below=1cm of retailer1] (dummy1) {}; 
	\node[mynode, left=of dummy1] (retailer3) {Кибернетика};  
	\node[mynode, right=of dummy1] (retailer4) {Теория игра};
	\node[below=2.5cm of retailer1] (dummy2) {}; 
	\node[mynode, left=of dummy2, align=left] (retailer5) {Теория \\ информации};  
	\node[mynode, right=of dummy2, align= left] (retailer6) {Теория \\ решений};
	\node[below=4.2cm of retailer1] (dummy3) {}; 
	\node[mynode, left=of dummy3, align=left] (retailer7) {Теория \\ графов};  
	\node[mynode, right=of dummy3, align= left] (retailer8) {Топология};
	\node[below=6cm of retailer1] (dummy4) {}; 
	\node[mynode, left=of dummy4, align=left] (retailer9) {Факторный \\ анализ};  
	\node[mynode, right=of dummy4, align= left] (retailer10) {ОТС (в узком \\ смысле)};
	
	\node[below =1cm of retailer2] (dummy5) {}; 
	
	\node[mynode, below =of retailer2, align=left] (retailer11) {Системотехника}; 
	\node[mynode, below =of retailer11, align=left] (retailer12) {Исслкдование \\ операций}; 
	\node[mynode, below =of retailer12, align=left] (retailer13) {Инженерная  \\ психология}; 
		
	\end{tikzpicture} 
	\medskip
	\caption{Структура теории систем} 
\end{figure}

	
	\section{Основные понятия теории систем.}
	
	Система - множество взаимосвязанных элементов, образующих целостность или органическое единство.
	
	\textbf{Признаки системы:}
		\begin{enumerate}
		\item Есть элементы и их довольно много
		\item Элементы как-то взаимосвязаны.
		\item Набор этих элементов образует нечно более сложное и несводимое к самим этим элементам.
	\end{enumerate}

	\textbf{Примеры систем:}
		\begin{enumerate}
		\item Система алгебраических или дифференциальных уравнений
		\item Замкнутая термодинамическая система
		\item Атом и его модели (в квантовой физике)
		\item Солнечная система, галактика, Вселенная
		\item Живая клетка, организм, популяция, биогеоценоз, биосфера Земли
		\item Завод, любое предприятие, макроэкономический объект(в экономике)
		\item Отдельно взятый человек (с точки зрения разных наук)
		\item Человеческий мозг (в анатомии и когнитивных науках)
		\item Группа людей, государство, человечество в целом (в социологии и истории)
	\end{enumerate}


	\textbf{Объект} - часть реального мира, выделаяемся как единое целое сейчас или в течение длительного промежутка времени.
	
	\textbf{Модель} - упрощенное представление объекта в человеческом восприятии или реальности.
	
	\textbf{Моделирование} - процесс создания модели на основе того или иного объекта.
	
	Важно понимать, что и объект и модель - системы, но это две разные модели. 
	
	При выделении системы обязательно проводится какая-то граница. То, что попало внутрь, образует систему.
	
	Внешняя среда, или системное окружение, - это всё, что осталось снаружи относительно рассматриваемой системы.
	
	Обычно во внешней среде находятся и другие системы. Поэтому взаимодействие данной системы со внешней средой обычно связано со взаимодействием и борьбой систем.
	
	Если данная системы является частью какой-то другой системы, то последняя называется надсистемой, а данная система - подсистемой.
	
	Довольно часто возникает целые иерархии из над- и подсистем различных уровней.
	
	При этом одна и та же система может выступать в качестве надсистемы( для систем нижних уровней) и подсистемой(для систем верхних уровней)
	
	\textbf{Элемент} - достаточно простая часть системы с однозначно определенными свойствами и функционированием, которую не целесообразно разбивать на более мелкие части в рамках решаемой задачи.
	
	\textbf{Связь} - специальные элементы, которые обеспечивают взаимодействие между остальными элементами(или подсистемами), а так же и с окружающей средой.
	
	\textbf{Структура системы} - устойчивое множество элементов и связей, которое сохраняется длительное время в системе.
	
	Существуют различные структуры со своими достоинствами и недостатками.
	
	Структурная перестройка - процесс, при котором сразу или постепенно меняется структура системы.
	
	\textbf{Функционирование системы} - деятельность системы во времени, обычно целенаправленная.
	
	Часто под функционированием системы понимают преобразование информации, вещества и/или энергии в какую-то другую информацию, вещество и/или энергии.
	
	\textbf{Вход} - всё то, что преобразуется системой в процессе функционирования.
	
	\textbf{Выход} - всё то, что получается в результате функционирования
	
	\textbf{Критерии функционирования} - признаки, по которым производится соответствие между реальным функционированием системы и желаемым(целевым)
	
	\textbf{Эффективность системы} - количественная мера качества работы системы, устанавливается с помощью критериев функционирования.
	
	\textbf{Оценка системы} - процесс установления критериев, а также из измерения, с последующим расчетом эффективности.
	
	\textbf{Теория систем} - наука, изучающая наиболее фундаментальные понятия, свойства и аспекты систем в целом.
	
	Такое изучение носит абстрактный характер, то есть, по возможности, стараются не отвлекаться на частную специфику систем одного происхождения.
	
	Результаты такого изучения стараются аккуратно перенести на системы различных типов и источников происхождения.
	
	Объектом исследования является не физическая реальность, а системы - реальные объекты или разработанные модели.
	
	\textbf{Достоинства:}
	
	-Абстрактные и максимально общий характер изучения различных объектов
	
	-установление междисциплинарных свойств и законов функционирования различных систем
	
	-Потенциально больная предсказательная сила
	
	\textbf{Недостатки:}
	
	-Трудности с изучением различных систем широкого спектра происхождения
	
	-Трудности прикладного использования полученных результатов и закономерностей
	
	-Не всегда очевидная интерпретация модельных свойств и закономерностей применительно к реальным объектам.
	

	
	\section{История систем и  системного анализа. Семинар 10 сент.}
	
	Само слово "система" появилось в древней Элладе.
	
	Системный анализ - набор методов для решения проблем в системах с управлением. Системы с управлением в основном социальные и экономические.
	
	Основными подходами системного анализа являются:
	
	1) Формальное мышление обеспечивает отображение элементов и законом их взаимодействия, традиционно базируется на математике.
	
	1) Гуманитарное познание формирует образ, целостность. Вершиной гуманитарного знания традиционно считается философия. 
	
	Н. Винер - основоположник кибернетики.
	
	1934 г. - семинар в Принцтоне
	
	Для обобщения дисциплин
	
	Л. Берталанфи(1901 - 1972 г.) - основоположник теории систем.
	
	А.А. Богданов (1873 - 1928г.) - основоположник тектологии. (был самым первым из трех учёных)
	
	В 80-х гг 20-го века возникла синергетика - наука о самоорганизации живых и неживых систем.
	
	Рождение понятие система в философии древней Греции в районе 5-го века до нашей эры.
	
	Некоторые системные идеи есть у таких философов как Птолемей, Анексагор и Демокрид. Демокрид - предложил систему атомов.
	
	Учение философов эпохи Возрождение(Европа 15 - 16 века). Николай Коперник и Галелео Галилей
	
	Пико Делла Мирандола - итальянский философ
	
	Иммануил Кант
	
	Фихте 
	
	Гегель 
	
	Карл Маркс и Фридрих Энгельс
	
	Богданов
	
	
		\section{Системные свойства.}
	
	\textbf{Холизм} - позиция в философии и науке по проблеме соотношения части и целого, исходящая из качественного своеобразия и приоритета целого по отношению к его частям.
	
	В широком смысле холизм представлет собой установку на учёт всех сторон рассматриваемого явления и критическое отношение к любому одностороннему подходу.
	
	Холизм в современной науке:
		\begin{itemize}
		\item Синтетическая теория эволюции Дж. Холдейна
		\item Гештальтпсихология
		\item Интегральная психология К. Уилбера
		\item Феноменология Э. Гуссерля
		\item Ряд направлений социальной философии (К. Маркс, Э. Дюркгейм, Н. Луман)
		\item  Современная филосифия науки (К. Поппер, Т. Кун, П. Фейербенд)
	\end{itemize}

	\textbf{Холизм и теория систем}
		\begin{itemize}
		\item Системы целесообразно рассматривать с позиции холизма
		\item Попытки рассмотрения систем с точки зрения редукционизма наталкиваются на ряд проблем
		\item  Возникает вопрос: что в системах такого особенного ? Почему их нужно рассматривать как-то иначе, нежели просто физические или механические явления?
		\item Причина в том, что системам присущи некоторые особые свойства - системные свойста
	\end{itemize}

	\textbf{Эмерджентность} - способность к возникновению у системы в целом новых свойств по сравнению со свойствами её элементов.
	
	"Системы в биологии, психологии и социологии являются настолько сложными, что их поведение кажется "новым", "эмерджентным": оно не может быть логически выведено из свойств отдельных элементов системы" Л. фон Берталанфи.
	
	\textbf{Примеры эмержентности:}
	\begin{itemize}
		\item Сложные технические устройства
		\item Процессор и компьютер
		\item Эффект толпы и другие социологические эффекты
		\item Социальные сети в интернете
		\item Биологические популяции
		\item Работа биологических органов и систем
		\item Работа человеческого мозга
		\item Эмерджентность в экономике: конкуренция и информационные стартапы
	\end{itemize}

	\textbf{Целостность}
		\begin{itemize}
		\item Целостность системы означает, что каждый элемент системы вносит вклад в реализацию целевой фунции системы
		\item Целостность системы обеспечивается путем формирования и поддержания связей между компонентами
		\item  Целостность и эмерджентность - интегративные свойства
		\item Наличие интегративныех свойств явлется одной из важнейших черт системы. Целостность проявляется в том, что системы обладает собственной целью.
	\end{itemize}

	\textbf{Организованность и функциональность}
	\begin{itemize}
		\item Организованность - свойство систем, заключающиеся в наличии сложной структуры.
		\item Функциональность - проявление определенных свойств при взаимодействии с внешней средой
		\item  Эти два свойства предопределяют сложный характер развития систем во времени и их эволюцию, связанную с развитием структуры и её перестройкой.
	\end{itemize}

	\textbf{Наличие поведения}
	\begin{itemize}
		\item Процесс целенаправленного изменения во времени состояния системы называется поведением
		\item В отличие от управления, когда изменение состояния системы достигается за счет внешних воздействий, поведение реализуется исключительно самой системой, исходя из собственных целей
		\item  Это позволяет отнести наличие поведения к системным свойствам
	\end{itemize}

	\textbf{Свойство роста(развития)}
		\begin{itemize}
		\item Одним из основополагающих атрибутов систмного подхода является недопустимость рассмотрения объекта вне его развития, под которым понимается необратимое, направленное, закономерное изменение материи и сознания.
		\item В результате развития возникает новое качество или состояние объекта
		\item  Поэтому важным частным случаем поведения систем является свойство их роста или развития
	\end{itemize}

	\textbf{Устойчивость}
	\begin{itemize}
		\item Поведение системы определяется характером реакции на внешние воздействия
		\item С этим связано ещё одно важное системное свойство - устойчивость
		\item \textbf{Устройчивость}(гомеостазис) - способность системы противостоять внешним возмущающим воздействиям
		\item  От неё зависит продолжительность жизни системы или успешность её работы
	\end{itemize}

	\textbf{Формы устойчивости}
	
	Для простых систем характерны:
	
	- Прочность
	
	- Сбалансированность
	
	- Регулируемость
	
	Для более сложных систем характерны:
	
	- \textbf{Надежность} - свойство сохраниения структуры систем, немностря на гибель отдельных её элементов с помощью их замены или дублирования
	
	- \textbf{Живучесть} - активное подавление вредных качество и воздействий со стророны окружающей среды
	
	- \textbf{Способность к адаптации} - изменение поведения или структуры с целью сохранения, улучшения или приобретение новых качеств в условиях изменения внешней среды
	
	Обязательным условияем возможности адаптации является наличие обратных связей
	
	\section{Классификация систем}
	
	\textbf{Классификация систем по содержанию }
	
	Системы - реальные и абстрактные
	
	Реальные(объективно существуют) - естественные и искусственные 
	
	Абстрактные(плод наших мыслей) 
	
	\begin{figure}[h]
		\centering
		\includegraphics[width=0.7\linewidth]{"Снимок экрана 2021-10-07 в 12.58.53"}
		\caption{}
		\label{fig:--2021-10-07--12}
	\end{figure}
	

	Естественные - существуют в природе
	
	Искусственные - созданы человеком
	
	Системы непосредственного отображения - отражают определенные аспекты реальных систем(пытаются отображать реальные системы)
	
	Генирализованные - носят обобщающий характер
	
	Логико-эвристические модели 
	
	\textbf{Классификация по взаимодействия со внешней средой:}
	
	1) Открытые
	
	2) Закрытые - не взаимодействуют с окружающей средой или взаимодействуют строго определенным образом
	
	3) Комбинированные - промежуточный тип в котором содержатся закрытые и открытые подсистемы
	
	\textbf{По пространственно-временным свойствам:} (невзаимоисключающие)
	
	1) Простые - не имеют сложной структуры состоят из небольшого числа элементов и связей между ними
	
	2) Сложные - большим число элементов и внутренних связей, неоднородностью , структурным разнообразием, выполненим сложной функции или даже нескольких сложных функций
	
	3) Большие - системы ненаблюдаемые с позиции одного наблюдателя во времени или пространстве
	
	\textbf{Классификация по характеру функций}
	
	1) Специальные - единственность назначения и узкая профессиональная специализация обслуживающего персонала 
	
	2) Многофункциональные - много функция на одной и той же структуре
	
	3) Универсальные - много функция на одной и той же структуре но эти функции могут быть сильно разнородны
	
	\textbf{Классификация по характеру развития}
	
	1) Стабильные - структуры и функции не меняются в течении её существования, с течением времени 
	
	2) Развивающиеся - с течением времени значительно меняют свои функции и структуру 
	
	\textbf{Классификация по характеру связей между элементами}
	
	1) Детерминированные -состояние системы однозначно определяется начальными значениями, состояние системы может быть предсказано в любой момент будущего.
	
	2) Стохастические(вероятностные) - изменения в них носят случайный, вероятностный характер, поэтому предсказать поведение такой системы в будущем невозможно совсем или пресказание тоже будет вероятностным
	
	\textbf{Классификация по характеру структуры управления }
	
	1) Централизованные - есть доминирующая подсистема/элемент который управляет всей системой
	
	2) Децентрализованные - несколько главных систем или элементов
	
	\textbf{Классификация по степени организованности}
	
	1) Хорошо организованные - можно легко определить элементы системы,их взаимосвязь, можно провести декомпозицию, описать проблемную ситуацию в формальном виде, исследовать систему аналитическими методами. Солнечная система, система уравнений
	
	2) Плохо организованные(диффузные) - плохо можно понять взаимосвязь элементов
	
	\textbf{Классификация по назначению}( только технические и социальные )
	
	1) Производящие - производят товары или услуги
	
	2) Управляющие - управляют преобразованием 
	
	3) Обслуживающие - занимаются организацией и управлением двух других систем
	
	\textbf{Классификация по сложности поведения}
	
	1) Автоматические - однозначно реагируют на  ограниченный набор внешних воздействий. Внутренняя организация таких систем сводится к обычному состоянию
	
	2) Решающие - имеют критерии различных реакция на внешнее воздействие, такие системы решают что им делать  в зависимости от этих критерий. Постоянство поддерживается заменой вышедшей из строя частей.
	
	3) Самоорганизующиеся - имеют гибкие критерии различия и гибкие реакции на внешние воздействия. Устойчивость структуры - постоянное самовоспроизводство. 
	
	4) Предвидящие - могут предвидеть внешние воздействия и заранее выработать критерии поведения.
	
	5) Превращающиеся - не связаны постоянством своей структуры 
	
	3) Самоорганизующиеся
	
	\section{Введение в теорию игр и принятия решений. Семинар 8 окт.}
	
	Теория игр - раздел прикладной матемитики
	
	Платёжная матрицы - прямоугольная матрица, где по строкам находятся наши стратегии, а по столбцам стратегии соперника, элемент $ a_{ij} $ - наш выигрыш при при выборе нами стратегии i, и выборе соперником j.
	
	Максимин(нижняя цена игры) - минимум по столбцам и максимум по строкам.
	
	Минимакс(верхняя цена игры) - максимум по столбцам и минимум по строкам.
	
	Если верхняя и нижняя цена игры совпадают, то это называется чистой ценой игры.
	
	Если верхняя и нижняя цена не совпадают
	
	\section{Системные представления.}
	
	Система - сложный объект для описания в силу нетривиального внутреннего строения и поведения
	
	Наличие системных свойств мешает применить к системам принцип редукционизма
	
	Многие системы функционируют и меняются в реальном времени, что усложняет их исследование
	
	При описании и исследовании систем приходится прибегать к моделированию, часто многовариантному.
	
	\textbf{Модель} - описание системы, отражающее определнную группу её свойств
	
	Модель сама является системой, обычно более простой, чем объект моделирования. Это упрощает её исследование и позваоляет осторожно перенести результаты на объект моделирования. В частности, предсказать поведение объекта и дать рекомендации по улучшению его структуры и функционированию.
	
	В процессе моделирования очень важным является четко опрделеить направление разработки модели - её контекст, точку зрения и цель.
	
	\textbf{Контекст} модели очерчивает  границы моделируемой системы и описывает её взаимосвязи с внешней средой
	
	\textbf{Цель} отражает причину создания модели и определяет её назначение. Все взаимодействия элементов модели рассматриваются с связи с достижением поставленной цели
	
	\textbf{Точка зрения} определяет, что будет рассматриваться и под каким углом. Необходимо помнить, что одна модель представляет одну точку зрения. Для моделирования системы с нескольких точек зрения используется несколько разных моделей.
	
	\textbf{Декомпозиция} - процесс последовательного выделения отдельных взаимосвязанных подсистем, которые, в свою очередь, так же могут быть декомпозированы. В качестве систем могут выступать не только материальные объекты, но и процессы, явления и понятия.
	
	Декомпозиция отличается от маханического разделения системы на части без учета системных свойств и взаимного влияния элементов и подсистем.
	
	Большинство систем могут быть декомпозированы на базовые представления подсистем. К ним относят: последовательное(каскадное) соединение элементов, параллельное соединание элементов, соединение с помощью обратной связи.
	
	Проблема проведения декомпозиции состоит в том, что в сложных системах отсутствует однозначное соответствие между законом функционирования подсистем и алгортмом, его реализующим. Поэтому осуществляется формирование нескольких вариантов декомпозиции системы. 
	
	\textbf{Системные представления:}
		\begin{itemize}
		\item Макроскопическое
		\item Теоретико-множественное
		\item Структурное
		\item Функциональное
		\item Информационное
		\item Процессуальное
	\end{itemize}

	\textbf{Макроскопическое представление} - понимание системы как нерасчленимого целого, взаимодействующего с внешней средой.
	
	С него обычно и начинают анализ в ситуации недостатка информации об исследуемой системе.
	
	Поскольку внешняя среда предствалена другими системами, то в центре рассмотрения оказывается взаимодействие исследуемой системы с другими системами.
	
	В зависимости от характера взаимодействия с другими системамти функционирование изучаемой системы можно расположить по возрастающему рангу следующим образом:
		\begin{itemize}
		\item пассивное существование
		\item материал для других систем
		\item обслуживание систем более высокого порядка
		\item противостояние другим системам(выживание)
		\item поглощение других систем(экспансия)
		\item преобразование других систем и сред(активная роль)
	\end{itemize}

	\textbf{Чёрный ящик} - термин, используемый дл обозначения системы, внутреннее устройство и механизм работы которой очень сложны, неизвестны или неважны в рамках данной задачи.
	
	\textbf{Теоретика-множественное представление} - рассмотрение системы как набора нескольких множеств и отношений на этих множествах
	
	При таком предствлении используется аппарат теории множеств и теории вероятностей, поскольку возникающие системы часто являются стохастическими
	
	Такое предствление удобно для объектов, докускающих формализацию и описываемых естественно устроенным набором параметров.	
	
	Каждому свойству объекта назначается переменная, с помощью которой суммируется изменение проявлений свойства
	
	Множеству наблюдаемых проявлений свойства ставится в соответствие множется значений переменной:
	
	$ D: S_i = [S_{i,j}, j = \{1, N\}] \rightarrow X_i = [X_{i,j}, j = \{1, N\}]$ , где $ S_i $ - i-ое свойство, $ X_i $ - переменная
	
	Процедура наблюдения свойств объекта включает базу и канал наблюдения.
	
	\textbf{Под базой наблюдения} понимается признаки различения одного проявления свойства от другого. Типовыми базами являются время, пространство, группа и их комбинации.
	
	\textbf{Параметр наблюдения} - операционное выражение базы.
	
	\textbf{Канал наблюдения} - процесс сопоставления значению параметра значения переменной.
	
	Необходимо различать четкие и нечеткий канал наблюдения.
	
	\textbf{Четкий канал} назначает одному значению параметра одно значение переменной. В этом случае система задаётся на чётком множестве значений переменных
	
	\textbf{В нечётном канале} наблюдения не существует однозначного решения о том, какое значение переменной назначить определённому значению параметра. Поэтому система задаётся в виде нечётких множеств состояний переменных.
	
	Формально система может быть представлена в виде множества: $ S = (X, T, R, Z) $, где X - множество переменных, T - множество параметров, R - отношения на множестве X и  T, Z - цель исследований.
	
	Отношения между переменными и параметрами здесь понимаются в самом широком смысле, включая ограничение, сцепление, соединение и т.д.
	
	На основе системы, формально заданной как набор множеств, в дальнейшем формально определяют:
	\begin{itemize}
		\item Структуру системы
		\item Полное множество состояний системы
		\item Функцию ограничения на полном множестве состояний(детерминированную или вероятностную)
		\item Меру нечеткости множества состояний системы( для стохастических систем)
		\item Меру сложности системы
		\item Методы упрощения системы и др.
	\end{itemize}

	\section{Структурное представление систем.}
	
	\textbf{Структурное (морфологическое) представление} дает представление о строении системы (морфология - наука о форме, строении). Глубина описания, уровень детализации, т.е. определние какие компоненты системы будут рассматриваться в качестве элементарных элементов, обусловливается целью и точкой зрения. 
	
	Подготовительным этапом для структурного описания является \textbf{структурный анализ}.
	
\textbf{Цели структурного анализа:}
		\begin{itemize}
		\item Разработка правил символического отображения систем
		\item Оценка качества структуры системы
		\item Изучение структурных свойств системы в целом и её подсистем
		\item Выработка заключения об оптимальности структуры системы и рекомендация по дальнейшему её совершенствованию.
	\end{itemize}

	Изучение морфологии системы начинается с \textbf{элементного состава}. Он может быть:
		\begin{itemize}
		\item \textbf{Гомогенным}(однотипные элементы) - сопутствуют как правило простота, надежность, избыточность и наличие скрытых возможностей, дополнительных резервов,
		\item \textbf{Гетерогенным}(разнотипные элементы) - специализированы, они экономичны и могут быть эффективными в узком диапазоне внешних условий, но быстро теряют эффективность вне этого диапазона.
		\item Смешанным
	\end{itemize}

	 По своему назначению различают элементы:
	 \begin{itemize}
	 \item \textbf{Информационные} - предназначены для приема, хранения, преобразования и передачи информации.
	 \item \textbf{Энергетические} - предназначены для изменения параметров энергетического потока. Поток входной энергии может поступать извне, либо от других элементов системы. Выходной поток направлен в другие элементы, либо в среду.
	 \item \textbf{Вещественные} - предназначены для преобразования материи. Процесс такого преобразования может быть механическим, химическим, физическим, биологическим
 \end{itemize}

	Структурные свойства системы существенно зависят от характера \textbf{связей} между элементами. Связь одновременно характеризует и строение (статику) и функционирование (динамику) системы. Связи обеспечивают возникновение и сохранение структуры и свойств системы.
	
	\textbf{Прямые  связи}  предназначены  для  заданной  функциональной передачи вещества, энергии, информации или их комбинаций — от одного элемента к другому в направлении основного процесса.
	
	\textbf{Обратные связи} выполняют осведомляющие функции, отражая изменение состояния системы в результате управляющего воздействия на нее. 
	
	Основные функции обратной связи:
	\begin{itemize}
		\item Противодействие тому, что делает сама система, когда она выходит за установленные пределы.
		\item Компенсация возмущений и поддержание состояния устойчивого равновесия системы
		\item Синтезирование внешних и внутренних возмущений, стремящихся вывести систему из состояния устойчивого равновесия, сведение этих возмущений к отклонениям одной или нескольких управляемых величин
		\item Выработка управляющих воздействий на объект управления по плохо формализуемому закону
	\end{itemize}

	\textbf{Детерминированная} (жесткая) связь - однозначно определяет причину и следствие, дает четко обусловленную формулу взаимодействия элементов.
	
	\textbf{Вероятностная} (гибкая) связь - определяет неявную, косвенную зависимость между элементами системы. Теория вероятности предлагает математический аппарат для исследования этих связей, называемый «корреляционными зависимостями».

	\textbf{Связь характеризуется: }
		\begin{itemize}
		\item Направлением - направленные и ненаправленные
		\item Силой - сильные и слабые
		\item Характером - подчинения, порождения, равноправные и связи управления.
	\end{itemize}

	По устойчивости отношений между элементами структуры делятся на:
		\begin{itemize}
		\item \textbf{Детерминированные} - отношения элементов либо постоянны, либо изменяются во времени по детерминированным законам
		\item \textbf{Вероятностные} - изменяются во времени по вероятностным законам
		\item \textbf{Хаотические} - характерны тем, что элементы образуют связь в соответствии с индивидуальными свойствами
	\end{itemize}

	\textbf{По характеру отношений между элементами} структуры делятся на:
			\begin{itemize}
				\item Многосвязные
				\item Иерархические
				\item Смешанные
	\end{itemize}

	\textbf{Координация} - выражает взаимодействие элементов одного уровня организации.


	\textbf{Субординация} - выражает взаимодействие элементов различных уровней организации, связанных с подчинением и управлением.
	
	\textbf{Иерархия} - это расположение частей целого в порядке от высшего к низшему. Иерархия определяет упорядоченность компонентов системы по степени важности.
	
	\textbf{Сильные иерархии} (иерархии типа "дерево") -  между уровнями иерархии структуры могут существовать взаимоотношения строгого подчинения компонент нижележащего уровня одному из компонент вышележащего уровня
	
	\textbf{Слабые иерархии} - могут быть связи в пределах одного уровня или нижлежащая компонента может подчиняться нескольким компонентам вышележащего уровня.
	
	\textbf{Свойства систем строгой иерархической структуры}:
	\begin{itemize}
		\item В системе имеется один главный управляющий элемент, который имеет не менее двух связей
		\item Имеются исполнительные элементы, каждый из которых имеет только одну связь с элементом вышележащего уровня
		\item Связь существует только между элементами, принадлежащим двум соседним уровням
	\end{itemize}

	\textbf{Свойства систем слабой иерархической структуры}:
	\begin{itemize}
		\item Связи не только между соседними уровнями иерархии
		\item Не все элементы промежуточных уровней имеют управляющие связи
	\end{itemize}
	
	\tikzstyle{level 1}=[level distance=1.5cm, sibling distance=2.5cm]
	\tikzstyle{level 2}=[level distance=1.5cm, sibling distance=1cm]
	
	\begin{figure}[h]
		\centering
		\begin{tikzpicture}
			\node {1}
			child { node {2}
				child { node {3}}
				child { node {3}}
			}	
			child { node {2}
				child { node {3}}
				child { node {3}}
			}
			child { node {2}
				child { node {3}}
				child { node {3}}
			}
	
			;
			
		\end{tikzpicture} 
		\medskip
		\caption{Граф строго иерархической структуры} 
	\end{figure}

	\begin{figure}[h]
		\centering
		\begin{tikzpicture}
			\node {1}
			child { node {2}
				child { node {3}}
				child { node {3}}
			}	
			child { node {2}
				child { node {3}}
				child { node {3}}
					child { node {3}}
			}
			child { node {3}
			}
			
			;
			
		\end{tikzpicture} 
		\medskip
		\caption{Граф нестрого иерархической структуры} 
	\end{figure}

	Для \textbf{неиерархических структур} не существует элементов, которые являются только управляющими или только исполнительными. Любой компонент взаимодействует более чем с одним компонентом
	
	Смешанные структуры представляют собой различные комбинации иерархических и неиерархических структур
	
	\textbf{В рамках структурного представления выделяют:}
	
	\underline{Состав}: 
	\begin{itemize}
		\item Гомогенный
		\item Гетерогенный
		\item Смешанный
		\item Неопределенный
	\end{itemize}

	\underline{Свойства элементов:}
		\begin{itemize}
		\item Информационные
		\item Энергетические
		\item Вещественные
		\item Неопределенные
	\end{itemize}

	\underline{Назначение  и характер связи:}
	\begin{itemize}
		\item Прямые
		\item Обратные
		\item Нейтральные
	\end{itemize}

	\underline{Устройчивость стуктуры:}
	\begin{itemize}
		\item Детерминированная
		\item Вероятностная
		\item Хаотическая
	\end{itemize}

	\underline{Топологии:}
\begin{itemize}
	\item Иерархические
	\item Многосвязые
	\item Смешанные
	\item Преобразующиеся
\end{itemize}
	
	Структурное представление строится по иерархическому принципу путем последовательной декомпозиции системы. 
	
	\textbf{Структурная схема} – совокупность блоков, осуществляющих некоторые функциональные преобразования, и связей между ними. Под блоком обычно понимают, особенно в технических системах, функционально законченное и оформленное в виде отдельного целого устройство.
	
	Блоки бывают \textbf{функциональные} и \textbf{логические} - позволяющие изменять характер функционирования в зависимости от того, выполняются или нет некоторые заранее заданные условия.
	
	\textbf{Достоинства структурных схем:}
	\begin{itemize}
		\item Наглядны
		\item Содержат информацию о большом числе структурных свойств системы
		\item Легко поддаются уточнению и конкретизации, в ходе которых не надо изменять всю схему, а достаточно заменить отдельные ее элементы структурными схемами, включающими не один, как раньше, а несколько взаимодействующих блоков
	\end{itemize}

 	\textbf{Недостатки структурных схем:}
 	\begin{itemize}
 		\item плохо описывают уже более-менее сложные системы
 		\item плохо поддаются формализации и системному анализу
 	\end{itemize}
 
 Отношения между элементами структуры могут быть представлены соответствующим \textbf{графом}. Это позволяет формализовать процесс исследования инвариантных во времени свойств систем и использовать хорошо развитый математический аппарат теории графов.
	
	\section{Функциональное представление систем}
	
	\textbf{Функциональное представление} или функциональная модель исходит из того, что всякая система выполняет некоторые функции.
	
	Способы функционального представления:
	\begin{itemize}
		\item \textbf{Алгоритмический} - функционирование системы описывается с помощью детерминированного алгоритма, результат алгоритма интерпретируется как основная функция системы
		\item \textbf{Аналитический} - числом функционирование системы описывается функционалом, зависящем от функций, описывающих внутренние процессы системы или  качественным функционалом(лучше - хуже, больше - меньше)
		\item Графический
		\item Табличный
		\item Посредством временных диаграмм функционирования - используются для технических, в частности человеко-машинных систем.
		\item Вербальный
	\end{itemize}

	Для временных диаграмм функционирования существует следующий порядок работы:
	\begin{enumerate}
		\item Выбирается некоторый интервал – длительность временного цикла
		\item Определяются параметры функционирования этих компонентов
		\item Составляются диаграммы функционирования их во времени
		\item Уточняются места переработки информации активными компонентами модели, называемыми в дальнейшем процессами системы
		\item Для каждого процесса большой системы определяется состав входной и выходной информации
		\item Уточняется состав управляющих параметров, влияющих на алгоритм функционирования процессов системы
		\item При необходимости выводятся уравнения и расчетные формулы
	\end{enumerate}

	В рамках каждого временного цикла проводится опрос параметров. Первичная обработка, контроль достоверности и выходов за уставки опрошенного параметра проводятся сразу же после приема значения параметра, в промежутке между запросом на измерение и приходом замеренного значения следующего параметра
	
	\textbf{Требования для представления:}
	\begin{itemize}
		\item Должно быть открытым и допускать возможность расширения (сужения) спектра функций, реализуемых системой
		\item Предусматривать возможность перехода от одного уровня рассмотрения к другому, т.е. обеспечивать построение виртуальных моделей систем любого уровня
	\end{itemize}
	
	\textbf{Графические способы функционального представления систем}
	
	Разновидности графического представления:
	\begin{itemize}
		\item Дерево функций системы
		\item Стандарт функционального моделирования IDEF0
	\end{itemize}

	\textbf{Дерево функций системы}
	
	Все функции системы могут быть разделены на 3 группы:
		\begin{itemize}
		\item \textbf{Целевая функция} - отражает назначение, сущность и смысл существования системы
		\item \textbf{Основные функции}  - представляют собой совокупность макрофункций, которые обеспечивают условия выполнения целевой функции
		\item \textbf{Вспомогательные функции}  - расширяют функциональные возможности системы, сферу их применения и способствуют улучшению показателей качества системы. Они обеспечивают условия выполнения основных функций.
	\end{itemize}
	
	\begin{figure}[h]
		\centering
		\includegraphics[width=0.6\linewidth]{"Снимок экрана 2021-11-08 в 22.03.44"}
		\caption{Представление объекта на языке функций}
		\label{fig:--2021-11-08--22}
	\end{figure}

	Дерево функций системы представляет декомпозицию функций системы и формируется с целью исследования функциональных возможностей системы. Производится анализ совокупности функций, реализуемых на различных уровнях иерархии системы. На базе дерева функций системы осуществляется формирование структуры системы на основе функциональных модулей.
	
	
	\textbf{Методология IDEF0}
	
	\textbf{IDEF0} - Function Modeling - методология функционального моделирования и графическая нотация, изначально предназначенная для формализации и представления бизнес-процессов.
	
	Отличительной особенностью IDEF0 является её акцент на соподчинённость объектов. В IDEF0 рассматриваются логические отношения между работами, а не их временная последовательность.
	
	\textbf{Основные элементы и понятия IDEF0}
	
	\textbf{Функциональный блок} -  графически изображается в виде прямоугольника и олицетворяет собой некоторую конкретную функцию в рамках рассматриваемой системы. По требованиям стандарта название каждого функционального блока должно быть сформулировано в глагольном наклонении.
	
	Пример: "обработать делать на станке" , "предать документы в отдел" .
	
	\tikzstyle{block} = [draw, fill=white, rectangle, 
	minimum height=3em, minimum width=6em]
	\tikzstyle{pinstyle} = [pin edge={<-,black}]
	\tikzstyle{output} = [coordinate]
	
	\begin{tikzpicture}[auto, node distance=2cm,>=latex']
		% We start by placing the blocks
		\node [block, pin={[pinstyle]above:Управление},pin={[pinstyle]left:Вход},pin={[pinstyle]below:Механизм},
		node distance=3cm, align=center] (system) {Функциональный \\ блок};
		\node [output, right of=system] (output) {};
		 \draw [->] (system) -- (output) node[midway,sloped, above right] {Выход};
		
	\end{tikzpicture}
	
	\textbf{Назначения сторон блока:}
		\begin{itemize}
		\item Верхняя сторона имеет значение “Управление” (Control)
		\item Левая сторона имеет значение “Вход” (Input)
		\item Правая сторона имеет значение “Выход” (Output)
		\item Нижняя сторона имеет значение “Механизм” (Mechanism)
	\end{itemize}

	Каждый функциональный блок в рамках единой рассматриваемой системы должен иметь свой уникальный идентификационный номер
	
	\textbf{Интерфейсная дуга} (Arrow) - отображает элемент системы, который обрабатывается функциональным блоком или оказывает иное влияние на функцию, отображенную данным функциональным блоком. Графически обозначается однонаправленной стрелкой. Каждая дуга должна иметь свое уникальное наименование
	
	\begin{figure}[h]
		\centering
		\includegraphics[width=0.6\linewidth]{"Снимок экрана 2021-11-08 в 22.28.56"}
		\caption{Дуги, как ограничивающие и уточняющие факторы Блока}
		\label{fig:--2021-11-08--22}
	\end{figure}
	
	В зависимости от того, к какой из сторон подходит данная интерфейсная дуга, она носит название “входящей”, “исходящей” или “управляющей”.
	
	Функциональный блок по требованиям стандарта должен иметь, по крайней мере, одну управляющую интерфейсную дугу и одну исходящую.
	
	\tikzstyle{block} = [draw, fill=white, rectangle, 
	minimum height=3em, minimum width=6em]
	
	\textbf{Типы взаимосвязей между блоками:}
	\begin{itemize}
		\item \textbf{Взаимосвязь по управлению}  когда выход одного Блока влияет на выполнение функции в другом Блоке (разработать инструкцию $ \to $ собрать)
		
		\begin{tikzpicture}[auto, node distance=2cm,>=latex']
			% We start by placing the blocks
			\node [block] (1) {1};
			\node [block, below  right =2 cm of 1] (2) {2};
			% We draw an edge between the controller and system block to 
			% calculate the coordinate u. We need it to place the measurement block. 
			\draw[->, rounded corners] (1.east) -| (2.north);
		\end{tikzpicture}
		
		
		\item \textbf{Взаимосвязь по входу} - когда выход одного Блока является входом для другого (изготовить подшипник $\rightarrow$ собрать машину)
		
		\begin{tikzpicture}[auto, node distance=2cm,>=latex']
			\node [block] (1) {1};
			\node [block, below  right =2 cm of 1] (2) {2};
			\draw[->, rounded corners] (1.east) -| (2,-2) |- (2.west);
		\end{tikzpicture}
		
		\item \textbf{Обратная связь по управлению} - когда выходы из одной функции влияют на выполнение других функций, выполнение которых в свою очередь влияет на выполнение исходной функции (изменение инструкции)
		
		\begin{tikzpicture}[auto, node distance=2cm,>=latex']
			\node [block] (1) {1};
			\node [block, below  right =2 cm of 1] (2) {2};
			\draw[->, rounded corners] (1.east) -| (2.north);
			\draw[->, rounded corners] (2.east) -| (5,1) -| (1.north);
			
		\end{tikzpicture}
		
		\item \textbf{Обратная связь по входу} - когда выход из одной функции является входом для другой функции, выход которой является для него входом (безотходное производство)
		
			\begin{tikzpicture}[auto, node distance=2cm,>=latex']
			\node [block] (1) {1};
			\node [block, below  right =2 cm of 1] (2) {2};
			\draw[->, rounded corners] (1.east) -| (2,-2) |- (2.west);			
			\draw[->, rounded corners] (2.east) -| (5,-3.5) -- (-1.5,-3.5) |- (1.west);
			
		\end{tikzpicture}
	
	\item 	\textbf{Взаимосвязь «выход-механизм»} - когда выход одной функции является механизмом для другой. Иначе говоря, выходная Дуга одного Блока является Дугой механизма для другого. 
		
		\begin{tikzpicture}[auto, node distance=2cm,>=latex']
			% We start by placing the blocks
			\node [block] (1) {1};
			\node [block, above  right =2 cm of 1] (2) {2};
			% We draw an edge between the controller and system block to 
			% calculate the coordinate u. We need it to place the measurement block. 
			\draw[->, rounded corners] (1.east) -| (2.south);
		\end{tikzpicture}
		
	\end{itemize}

	\textbf{Декомпозиция} - применяется для разбиения сложного процесса на составляющие его функции. Детализация процесса определяется непосредственно разработчиков
	
	Декомпозиция позволяет постепенно и структурированно представлять модель системы в виде иерархической структуры отдельных диаграмм, что делает ее менее перегруженной и легко усваиваемой. 
	
	\textbf{Глоссарий} - список определений для ключевых слов, фраз и аббревиатур, связанных с блоками, дугами или с моделью в целом. Представление сущности данного элемента.
	
	Разработка IDEF0 Диаграмм начинается с построения самого верхнего уровня иерархии (А-0) — одного Блока и интерфейсных Дуг, описывающих внешние связи рассматриваемой системы. Имя функции, записываемое в Блоке 0, является целевой функцией системы с принятой точки зрения и цели построения модели.
	
	\textbf{Ограничения сложности IDEF0-моделей:}
	\begin{itemize}
		\item Ограничение количества функциональных блоков на диаграмме тремя-шестью. Верхний предел заставляет разработчика использовать иерархии при описании сложных предметов, а нижний предел гарантирует, что на соответствующей диаграмме достаточно деталей, чтобы оправдать ее создание
		\item Ограничение количества подходящих к одному функциональному блоку (выходящих из одного функционального блока) интерфейсных дуг четырьмя
	\end{itemize}

	\textbf{Стадии процесса разработки моделей:}
	\begin{enumerate}
		\item \textbf{Предварительное создание модели} группой специалистов, относящихся к различным сферам деятельности предприятия. Эта группа в терминах IDEF0 называется авторами (Authors). На основе имеющихся положений, документов и результатов опросов создается черновик (Model Draft) модели. 
		\item \textbf{Сбор отзывов}. Распространение черновика для рассмотрения, согласований и комментариев. На этой стадии происходит обсуждение черновика модели с широким спектром компетентных лиц (читателей) на предприятии. Этот цикл продолжается до тех пор, пока авторы и читатели не придут к единому мнению.
		\item \textbf{Официальное утверждение модели}. Утверждение согласованной модели происходит руководителем рабочей группы в том случае, если у авторов модели и читателей отсутствуют разногласия по поводу ее адекватности. Окончательная модель представляет собой согласованное представление о предприятии (системе) с заданной точки зрения и для заданной цели
	\end{enumerate}

	\textbf{Пример}
	
	Отдел контроля, созданный для оценки эффективности управления и функционирования библиотеки.
	
	\begin{figure}[h!]
		\centering
		\includegraphics[width=0.7\linewidth]{"Снимок экрана 2021-11-09 в 13.28.16"}
		\caption{Диаграмма верхнего уровня А-0, отражающая целевую функцию системы}
		\label{fig:--2021-11-09--13}
	\end{figure}
	
	\begin{figure}[h!]
		\centering
		\includegraphics[width=0.7\linewidth]{"Снимок экрана 2021-11-09 в 13.28.25"}
		\caption{Диаграмма А0, отражающая декомпозицию целевой функции на основные функции А1, А2, А3}
		\label{fig:--2021-11-09--13}
	\end{figure}
	
	\begin{figure}[h!]
		\centering
		\includegraphics[width=0.5\linewidth]{"Снимок экрана 2021-11-09 в 13.28.34"}
		\caption{Декомпозиция блока А1}
		\label{fig:--2021-11-09--13}
	\end{figure}

	\begin{figure}[h!]
		\centering
		\includegraphics[width=0.7\linewidth]{"Снимок экрана 2021-11-09 в 13.28.49"}
		\caption{Декомпозиция блока А2}
		\label{fig:--2021-11-09--13}
	\end{figure}
	
	\begin{figure}[h!]
		\centering
		\includegraphics[width=0.7\linewidth]{"Снимок экрана 2021-11-09 в 13.29.23"}
		\caption{Декомпозиция блока А3}
		\label{fig:--2021-11-09--13}
	\end{figure}
	
	\newpage
	
	\section{Информационное представление систем}
	
	\textbf{Информационное описание} - дает представление об информационных потоках в системе.
	
	\textbf{Информационное моделирование} - нацелено на оптимизацию информационных потоков
	
	Значения термина \textbf{информация}:
	\begin{itemize}
		\item Совокупность каких либо сведений или знаний
		\item Сведения, являющиеся объектом хранения, передачи и переработки
		\item Сведения, сигналы об окружающем мире, которые воспринимают организмы в процессе жизнедеятельности
		\item В биологии - совокупность химически закодированных сигналов, передающихся от одного живого организма другому или от одних клеток, тканей, органов другим в процессе развития особи
	\end{itemize}

	В теории информации рассматриваются следующие уровни:
		\begin{itemize}
		\item \textbf{Синтаксический} - рассматриваются внутренние свойства текста: отношение между знаками алфавита, отражающие структуру данной знаковой системы
		\item \textbf{Сементический} - анализируются отношения между знаками и обозначаемыми ими предметами, действиями, качествами, т.е. смысловое содержанием текста
		\item \textbf{Прагматический} - рассматриваются отношения между текстом и тем, кто его использует, т.е. ценность информации для потребителя
	\end{itemize}

	\textbf{Ценность информации} выражается через приращение вероятности достижения цели:
	
	$ p_0 $ - вероятность достижения цели до получения информации
	
	$ p_1 $ - вероятность достижения цели после получения информации
	
	Величина ценности информации описывается следующей формулой( формулой Харкевича):
	
	$ I_0 = \log_2 \frac{p_1}{p_0} $
	
	\textbf{Граф информационного описания} - используется в социально-технических и экономических системах. Исследуемая система представляется иерархической структурой: на нижних уровнях - участки технологического процесса, а на верхних - узлы управления.
	
	Информационная структура управления сложной системой является графом с преимущественно упорядоченными вершинами, который в частном случае сводится к иерархическому дереву.
	
	\textbf{Первый информационный уровень} - это уровень непосредственного управления технологическими операциями. На следующих уровнях располагаются производственно-технологические подразделения и предприятия.
	
	\textbf{Формы информации в каналах связи системы:}
	\begin{itemize}
		\item \textbf{Осведомляющая} – движущаяся преимущественно от объектов управления к соответствующим узлам управления
		\item \textbf{Управляющая} – движется в обратном направлении и содержит указания, директивы и т.п.
		\item \textbf{Преобразующая} – определяет закономерности поведения узла управления и алгоритмы его функционирования
	\end{itemize}

	\textbf{Узлы управления} преобразуют осведомляющую информацию в управляющую с помощью преобразующей информации, заключенной в структуре и алгоритмах узла управления
	
	Чем меньше требуется информации от вышестоящих узлов для формирования информации управления в некотором i-м узле, тем более автономен этот узел.
	
	\textbf{Качественные параметры информационных потоков:}
	\begin{itemize}
		\item Общее время реагирования
		\item Интенсивность
		\item Избыточность
		\item Дублирование
		\item Нестабильность
		\item Погрешность
		\item Формы представления
	\end{itemize}

	\textbf{Количественные параметры информационных потоков в экономических системах:}
	\begin{itemize}
		\item Коэффициент трансформации или сжатия
		\item Коэффициент комплексности
		\item Коэффициент стабильности
	\end{itemize}

	\textbf{Для более формальных систем (математических, физических) используется понятие энтропии.}
	\begin{itemize}
		\item Термодинамическая энтропия - термодинамическая функция, характеризующая меру необратимой диссипации энергии в ней
		\item В статистической физике - характеризует вероятность осуществления некоторого макроскопического состояния системы
		\item В математической статистике - мера неопределённости распределения вероятностей
		\item Информационная энтропия - в теории информации мера неопределённости источника сообщений, определяемая вероятностями появления тех или иных символов при их передаче
		\item Дифференциальная энтропия - формальное обобщение понятия энтропии для непрерывных распределений
		\item Энтропия отражения - часть информации о дискретной системе, которая не воспроизводится при отражении системы через совокупность своих частей
		\item Энтропия в теории управления - мера неопределённости состояния или поведения системы в данных условиях
		\item Энтропия динамической системы - в теории динамических систем мера хаотичности в поведении траекторий системы
	\end{itemize}

Больше беспорядка = большая энтропия

\textbf{Негэнтропия} – мера порядка, организованности.

\textbf{Информация} (в математике и кибернетике) – количественная мера устранения энтропии (неопределенности), мера организации системы.

\textbf{Организованность, упорядоченность системы} – способность предопределять свою перспективу, свое будущее. Чем беспорядочнее система, тем больше зависит ее перспектива от случайных факторов (внутренних и внешних). 

Мера организованности - потенциальная мера предсказуемости будущего системы, количественная характеристика возможности предвидения состояния системы. 

\textbf{Информация об организации системы} – это количественная характеристика возможности предвидения ее состояния (поведения) на соответствующем уровне детализации системы.

\textbf{Информация о среде} - количественная характеристика возможности предвидения воздействия среды.


При информационном описании технических процессов и технических систем принято использовать понятие \textbf{количество разнообразия}.

В общем случае объект наблюдения  $ А $ может с некоторой вероятностью находиться в одном из $ k  $различных состояний.

Меру неопределенности системы $ H(p) $ или энтропию вычисляют по формуле Шеннона:

$ H(p) = \sum_{i=1}^{k}p_i \cdot \log_2 \frac{1}{p_i}$ , где $ p_i $ - вероятность i-го состояния система. 

Информация противоположна энтропии, являющейся мерой неопределенности состояний изучаемого объекта. Информация ограничивает разнообразие, снижает энтропию, устраняет неопределенность частично или полностью. По Шеннону, величина информации о событии равна уменьшению энтропии в системе под действием некоторого события.

\textbf{Результатом информационного описания системы, в общем случае, является:}
\begin{itemize}
	\item Состав информационных элементов
	\item Состав и структура информационных потоков между элементами
	\item Количество и формат информации, поступающей (исходящей) в (из) информационных элементов
	\item Алгоритмы преобразования информации в информационных элементах
\end{itemize}



\end{document}